Door de hoge dichtheid van servers per vierkante meter, de vaak grote en zware systemen voor koeling en UPS-systemen maken dat de vloerbelasting van een serverruimte vaak de standaard gebouwontwerp specificaties overstijgen. Als een standaard ruimte ingericht gaat worden als serverruimte moeten er vaak aanpassingen gedaan worden aan de constructie om de draagkracht te versterken. Het is handiger als bij de bouw al rekening is gehouden met de toekomstige inrichting. Bij de bouw van grote datacenters wordt dit dan ook gedaan.

Veel datacenters hebben verhoogde vloeren zodat een luchtstroom mogelijk is die gebruikt kan worden bij de koeling van de kasten. De ruimte onder de vloer kan ook gebruikt worden om kabels weg te werken. De verhoogde vloeren bestaan uit een geaarde staalconstructie die zorgt dat elektrostatische energie afgevoerd wordt. Het gebruik van anti-statische-polsbandjes is in serverruimtes dan ook meestal niet nodig.

Op de staalcontructie liggen roosters en tegels (60x60 cm). De dienen ter doorlating van de luchtstroom en de tegels blokkeren deze juist. Door roosters en tegels op strategische plaatsen te plaatsen kan er gezorgd worden voor een optimale luchtstroom in de ruimte.
