Naast de controle van de temperatuur sensoren, ter controle van de algehele temperatuur in het datacenter, is van belang om ook de racks te controleren op hot-spots. Een hot-spot is een plek in de serverruimte die meer warmte genereert dan een andere. Dit kan gebeuren omdat in bijvoorbeeld een rekencluster een aantal machines zijn toegewezen aan een rekenintensieve taak. Op deze manier kan een, tijdelijke, plek in de serverruimte ontstaan die extra warm wordt. Als de extra warmte van korte duur is kan het meestal niet zo veel kwaad, als het structureel is dan zullen er maatregelen genomen moeten worden.


