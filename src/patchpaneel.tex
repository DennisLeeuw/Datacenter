Patchpanelen zijn passieve componenten, dat betekent dat ze niet iets met het signaal doen. Patchpanelen worden veelal in serverkasten gebruikt om de servers met patchkabels te koppelen aan de bekabeling die naar de serverruimte switches lopen. De IEEE 802.3 ethernet standaard zegt dat de maximale lengte van een ethernet kabel 100 meter is waarbij er twee keer 5 meter patch kabel gebruikt mag worden en dat de overige 90 meter moet bestaan uit solide koper bekabeling. Deze solide koperbekabeling wordt dan ook vaak gebruikt als bekabeling tussen een patchpaneel en de switch waarbij binnen de kast gebruik gemaakt wordt patchkabels die bestaan uit stranded (getwijnde) bekabeling.

Patchpanelen kunnen bovenin of onderin de kast gemonteerd worden, of in het midden van de kast. Het voordeel van de montage in het midden van de kast is er minder verschillende kabellengtes nodig zijn om de servers aan te sluiten op de patchpanelen en dat de kabelbomen aan de zijkanten van de kast kleiner zullen zijn omdat de ene helft omhoog loopt en de andere helft maar beneden. Dit helpt in het overzichtelijk en netjes houden van de bekabeling in de kast.

Patchpanel montage:
\url{https://spl-play.learningcloud.me/admin/sprints/5e3b1e8e79a83f25dd79f792/}

Patchbekabeling:
\url{https://spl-play.learningcloud.me/admin/sprints/5e3b1eea32eea20fcd021bb7/}

