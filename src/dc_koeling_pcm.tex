PCM\index{PCM} staat voor Phase Change Material\index{Phase Change Material}. De techniek gebruikt de overgang van vast naar vloeibaar en van vloeibaar naar vast om kou of warmte op te slaan en werkt dus feitelijk als een thermische batterij.

De PCM techniek maakt gebruik van de buitenlucht en het feit dat de temperatuur 's nachts lager is dan overdag. Een anorganische stof wordt door menging met zouten zo ingeregeld dat deze vanaf een bepaalde temperatuur stolt of smelt. De warmte uit de serverruimte zal overdag de stof doen smelten hierdoor kan de PCM grote hoeveelheden warmte opnemen. Als de buitenlucht beneden de ingestelde temperatuur komt dan kan de buitenlucht de opgeslagen warmte afvoeren waardoor de stof weer stolt.

Deze techniek gebruikt aanmerkelijk minder stroom dan een traditionele airco-koeling. Fabrikanten claimen tot 90\% vermindering van het energieverbruik.
