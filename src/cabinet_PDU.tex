Het aansluiten van de servers gebeurt zoveel mogelijk in het cabinet, zodat de kast een gesloten geheel vormt. De spanning voor de servers wordt aangeboden via een PDU (Power Distribution Unit). Een PDU is vergelijkbaar met een verdeeldoos in het dagelijksgebruik. De aansluingen zijn vaak anders. De meest gebruikte aansluit stekkers zijn de C13/C14 stekkers:
\begin{figure}
\end{figure}

Een extra functie die je tegen kan komen in de PDUs is de mogelijkheid om elke uitgang te schakelen via bijvoorbeeld het netwerk (switched PDU). Via een remote protocol zoals bijvoorbeeld een web-interface of ssh kan je inloggen op de PDU en de spanning naar een socket schakelen. Op deze manier kan je een server die niet meer reageert op afstand herstarten in de hoop dat deze na een herstart weer op afstand bedient kan worden. Zo niet dan moet je toch naar de serverruimte lopen.

Een andere functie die je op PDUs kan aantreffen is een verbruiksmeter (metered PDU). Sommige PDUs meter alleen het totale verbruik van de aangesloten apparaten, andere varianten kunnen per socket meten hoeveel het verbruik is. Op deze manier kan je bijvoorbeeld het verbruik doorbelasten naar de gebruiker.
