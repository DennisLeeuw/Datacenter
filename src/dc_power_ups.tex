\index{UPS}\index{Uniterruptable Power Supply}De functie van de UPS (Uniterruptable Power Supply) is tweeledig. De eerste taak is het opvangen van kleine pieken en dalen in de elektriciteitsvoorziening. Computer apparatuur is over het algemeen gevoelig voor spanningswisselingen. Door de aanvoer eerst door een UPS systeem te laten gaan werken zorgen de batterijen van de UPS voor demping van pieken of opvulling van dalen waardoor er een veel gelijkmatigere aanvoer van vermogen is. De tweede functie van de UPS is het opvangen van de periode tussen het uitvallen van het elektriciteitsnet en het starten van de generator. Over het algemeen moet een UPS systeem het gehele datacentrum voor ongeveer 5 minuten van het volledige vermogen kunnen voorzien, zodat de generator de tijd heeft om meerdere keren te starten. Ook hier moet rekening gehouden worden met het feit dat de generator misschien niet meteen de eerste keer start.

UPS systemen zijn traditioneel gebouwd met batterijen, modernere varianten hebben een vliegwiel.

\url{https://download.schneider-electric.com/files?p_File_Name=DBOY-77FNCT_R2_EN.pdf&p_Doc_Ref=SPD_DBOY-77FNCT_EN}
