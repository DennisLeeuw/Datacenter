\index{Voeding!Hoogspanning}Het aansluiten van een datacentrum op het elektriciteitsnet in Nederland gebeurd meestal met 10 kVac of 20 kVac aansluitingen. Transformatoren zetten de aangeleverde hoogspanning om in 400 Vac welke de binnenkomende spanning is voor het datacentrum. Dit is de ingangsspanning voor de Main Distribution Boards (MDBs)\index{MDB}\index{Main Distribution Board}. MDBs zijn panelen of kasten die de zekeringen, schakelaars en aardlekzekeringen bevatten. De MDBs distribueren de spanning naar de verschillende onderdelen zoals UPS (Uninterrupted Power Supply)\index{UPS}\index{Uninterruptable Power Supply} systemen. Uiteindelijk wordt er dan 240 Vac naar de 19"-kasten gestuurd.

Het starten van de generator(en) is meestal een taak van de MDBs, deze doen dat zodra ze het wegvallen van het elektriciteitsnet waarnemen.

