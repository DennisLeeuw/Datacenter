Bij adiabatische koeling\index{adiabatische koeling} wordt de warme lucht uit de serverruimte gekoeld door koelers in de buitenlucht te besprenkelen met water waardoor verdamping optreedt. Dit kost natuurlijk water wat niet erg milieu vriendelijk is.

Een alternatief is de indirecte adiabatische koeling\index{indirecte adiabatische koeling}\index{adiabatische koeling!indirect}. Via een warmte wisselaar wordt de warme lucht direct gekoeld aan de buitenlucht, als de buitenlucht te warm is om voldoende te koelen dat wordt er teruggevallen op het gebruik van de verdamping van water voor de koeling. Voordeel van dit systeem is dat er geen buitenlucht in het datacenter terecht komt.
