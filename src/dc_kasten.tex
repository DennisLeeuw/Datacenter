\index{Kasten}\index{Cabinets}Een serverruimte kast kan bestaan uit een rack waarin de servers hangen met een voordeur en achterdeur en twee zijpanelen. Als er meerdere kasten naast elkaar staan kunnen de zijpanelen verwijderd worden zodat de kasten op elkaar aangesloten kunnen worden. Soms worden om de luchtstroom door een kast te bevorderen, zogenaamde schoorsteenwerking, de zijplaten niet verwijderd. Dit hangt erg van de manier van koelen af. De koeling kan van beneden naar boven door een kast lopen of van voor naar achter.

Kasten kunnen in hoogte verschillen. De meeste kasten zijn tussen de 32 en 54 U hoog. De U is een standaard eenheid van 1,75 inch. De breedte van de servers in de kasten is altijd 19 inch. De daadwerkelijk breedte van een kast hangt af van de extra ruimte die aanwezig is om bijvoorbeeld kabels netjes te kunnen wegwerken.

In de kast worden de servers en andere compenten met schroeven bevestigd aan flenzen die aan de zijkant van de kast zitten. Een PDU\index{PDU} (Power Distribution Unit\index{Power Distribution Unit}) of Power Strip\index{Power strip} zorgt voor de stroomvoorziening in de kast. PDUs kunnen 1 of meer Us innemen in de kast of ze kunnen aan de zijkant van de kast gemonteerd zijn.

Om te voorkomen dat de luchtstroom door de kast beperkt wordt zijn er verschillende kabelmanagement systemen verwerkt in een kast. Er zijn de horizontale kabelbegeleiders en er zijn vertikale kabelgeleiders. Kabels worden meestal met velcro-strips samen gebonden tot kabelbundels zodat de kast een net en verzorgt uiterlijk heeft.

\url{https://www.youtube.com/watch?v=TCiZjB9ZXgI}
