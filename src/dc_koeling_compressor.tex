\index{Compressor}\index{Airco}\index{Koeling!Airco}Compressor koeling ook wel de airco (air compressor) genoemd is de klassieke manier om datacenters te koelen. Het systeem werkt op basis van water. Warm water wordt naar een unit gebracht die buiten hangt of staat. De koude lucht koelt het water af en het koude water wordt weer naar binnen gepompt om de warme lucht uit de serverruimte te koelen. Als het water niet voldoende afkoelt aan de buitenlucht worden er compressorkoelers gebruikt om het water verder af te koelen.

Het nadeel van dit systeem is dat er dure compressors in zitten en pompen en dat het geheel zelf behoorlijk wat stroom verbruikt. Daarnaast geldt ook hier, hoe meer onderdelen hoe meer kans op defecten.
