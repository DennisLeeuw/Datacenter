\index{UPS!Batterijen}\index{Batterijen}Traditioneel gebruiken de meeste uninterruptable power supply systemen batterijen. Batterijen kunnen relatief makkelijk veel energie opslaan, het nadeel is dat ze veel onderhoud vergen en regelmatig gecontrolleerd moeten worden op hun functioneren.

Batterijen zijn DC (Direct Current) spanningscomponenten, de aanvoer is AC (Alternating Current) en ook de levering aan de cabinetten is een wisselspanning. Om als buffer te kunnen dienen moet eerst de aanvoer omgezet worden in DC en daar na moet de DC spanning weer omgevormd worden naar AC. De technieken voor een energiezuinige UPS halen een rendement van rond de 95\% of iets meer. Er blijven dus verliezen optreden en dus ontstaat er ook warmte.

\url{https://www.youtube.com/watch?v=UMtftcKACRA}


