\index{xNy}\index{Redundantie!xNy}Een variant op de 2N oplossing is een oplossing waarbij de lasten verdeeld worden. Stel we hebben een serverrack dat 1500 Watt nodig heeft bij totale belasting. Als we dat redundant willen maken met 2(N+1) betekent dat dat we 2x 1500 Watt UPS nodig hebben voor N+1 en dat moeten we verdubbelen voor 2x N+1, dus we hebben 4x een 1500 Watt UPS systeem nodig.

Door de lasten te verdelen kunnen we een soortgelijke bescherming krijgen tegen lagere kosten. Door de last te verdelen over bijvoorbeeld 3 systemen van 750 Watt zouden we bij uitval van 1 systeem een capaciteit van 1500 Watt over. We hebben dan dus 3 systemen, kleinere en goedkopere systemen, waarvan er 1 volledig mag uitvallen (2 blijven er werken). Dat noemen we \index{Redundantie!3N/2}3N/2. Voor de berekening van de capaciteit per systeem nemen we de totale capaciteit die we nodig hebben (N) en delen dat door het aantal systemen dat moet blijven functioneren (y). In dit voorbeeld dus 1500/2 = 750 Watt. Voor de redundatie hebben we 1 systeem nodig en daarmee komt het totaal aantal systemen (x) op 3.

Met \index{Redundantie!4N/3}4N/3 hebben we 4 systemen waarvan er 1 mag uitvallen. Uit ons vorige voorbeeld zou dat betekenen dat we 4 UPSen van 500 Watt (1500/3) neerzetten waarvan er 1 mag uitvallen zodat we dan nog steeds 1500 Watt over houden. 4N/2 geeft ons 4 UPSen van 750 watt waarvan er 2 mogen uitvallen.
