\index{Bekabeling}De bekabeling in een datacenter bestaat uit lange koperkabels, tot 90 meter, of fiber kabels die de verschillende componenten met elkaar verbinden. Deze zelfde kabeltypes worden ook gebruikt om verschillende datacentra onderling te koppelen. We spreken hier van de horizontale bekabeling door gebouwen, het zijn ook de kabels die lopen vanaf de switches naar de wall-oulets in de verschillende kantoren.

Naast deze horizontale bekabeling zijn er de patchkabels die switches of patchpanelen verbinden met servers of patchpanelen met de de poorten van switches. Het zijn de flexibele kabels die makkelijk verlegt kunnen worden.

De kabels in een datacentrum kunnen worden weggewerkt onder de vloer, boven een systeemplafond of in metale kabelgoten die aan het plafond hangen.
