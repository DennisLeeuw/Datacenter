De meeste kasten hebben in de kast een plek waar de servers aangesloten kunnen worden op het netwerk. De traditionele manier is dat een kast voorzien is van een patchpaneel. In het volgende hoofdstuk zullen we ook ander oplossingen bekijken. De servers worden met zogenaamde patchkabels (flexibele kabels) aangesloten op het patchpaneel. Om de koeling in de kast niet te veel te verstoren is het van belang dat de kabels op een ordelijke manier in de kast verwerkt worden en het geen spaghetti wordt.

Om de kabels netjes weg te werken zitten er horizontale kabel begeleiders in de kast om de bekabeling van de servers naar de zijkant te brengen en er zijn vertikale kabelbegeleiders die de kabels begeleiden van boven naar beneden of van beneden naar boven.

De patchpanelen zijn bekabeld met stugge (solide) kabels die de kast verbindt met de rest van de serverruimte.
