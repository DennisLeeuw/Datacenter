Een datacenter is erop ingericht om zoveel mogelijk rekencapaciteit (servers) kwijt te kunnen op een zo efficient mogelijke manier. De meest efficiente manier van inrichten is door standaardisatie. Desktops, laptops, mini- en midi-towers door elkaar heen levert geen efficiente inrichting op. De keuze is gemaakt om servers te maken met een breedte van 19" en een hoogte die moet voldoen aan 1U (1,75 inch, 44,45 mm) of een veelvoud daarvan.

Deze servers hangen in kasten die we racks of cabinetten noemen. De breedte van de kasten is gestandaardiseerd op de breedte van de servers en deze worden dan ook 19"-kasten, 19"-racks of 19"-cabinets genoemd.

In de kasten zit een voorziening voor het aansluiten van de servers op de spanning. Deze aansluitingen noemen we de power-strips. Om te zorgen dat dipjes en pieken gefilterd worden en een kortstondige uitval van de spanning opgevangen wordt zijn datacenters uitgerust met UPS-systemen. UPS staat voor Uninterruptable Power Supply en is een voorziening die ervoor zorgt dat bij uitval van de hoofd elektriciteitsvoorziening het datacenter toch van spanning en stroom voorzien blijft.

Binnen een datacenter vinden we naast de servers natuurlijk ook voorzieningen voor netwerkaansluitingen zoals patchpanelen, switches en routers en hele bossen aan netwerkkabels die op een gestructureerde manier weggewerkt moeten worden.

Omdat datacenters afgesloten ruimtes zijn met vaak beperkte toegang voor een paar bevoegden en zo ingericht zijn dat er zoveel mogelijk servers in passen is de warmte die door die server gecre\"eerd wordt een probleem. Koeling is dan ook een van meest cruciale onderdelen in het ontwerp van een datacentrum.
