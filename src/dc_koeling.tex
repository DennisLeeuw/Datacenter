Vroeger was het grootste probleem in een datacenter de ruimte, het snel groeiende aantal servers stuite op een ruimte probleem. Met de opkomst van virtualisatie is dat probleem getackeld. De virtualisatie cre\"eerde echter wel een ander probleem: koeling. 10 servers die hoofdzakelijk uit hun neus staan te peuteren verbruiken stroom, maar cre\"eeren nauwelijks warmte. E\'en server die stevig staat te stampen om 10 virtuele machines in de lucht te houden gebruikt minder stroom dan de tien bij elkaar opgeteld, maar zorgt voor meer warmte dan de 10 machines bij elkaar.

De warmte die veroorzaakt wordt door de servers in het datacenter moet worden afgevoerd. Traditioneel gebeurde dat met airco's, maar met de ontwikkeling van groenere datacentra wordt er steeds vaker voor andere technieken gekozen, zoals het koelen met buitenlucht of het koelen met vloeistoffen.

