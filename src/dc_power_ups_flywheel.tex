\index{UPS!Vliegwiel}\index{Vliegwiel}Het vliegwiel is geen nieuwe techniek, maar wel als functie in het datacenter. Veel serverruimtes maken nog gebruik van batterijen om voor langere tijd energie op te slaan. Een vliegwiel maakt gebruik van de traagheid van massa. De inkomende energie van het elektriciteitsnet zorgt ervoor dat een zware schijf rondgedraaid wordt, deze schijf is ook verbonden met een generator. Als de netspanning wegvalt zal het draaiende gewicht de generator laten doordraaien en zo kan ongeveer 15 seconden tot een minuut aan spanningsdip opgevangen worden. In die 15 seconden moet een dieselgenerator of gasturbine gestart worden die de wegvallende netspanning opvangt.

\url{https://www.youtube.com/watch?v=kQirOFEygJQ}
