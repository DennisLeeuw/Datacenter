Een redundant systeem heeft twee aanvoerleidingen vanuit het elektriciteitsnet, het liefst vanuit een verschillende richting of van verschillende leveranciers. Dit zorgt ervoor dat er spanning blijft ook als \'e\'en aanvoerlijn uitvalt. Daarmee zijn we nog niet veilig want ook het totale elektriciteitsnet kan uitvallen en dan zit het datacentrum alsnog zonder spanning. Om dat op te kunnen vangen worden er vaak generatoren geplaatst bij een datacentrum. De generatoren werken op diesel of gas en wekken elektriciteit op dat voldoende is om een datacentrum voor bijvoorbeeld 24 of 48 uur van spanning te voorzien. Het opstarten van generatoren kost echter tijd, we laten ze niet dag en nacht draaien dat zou te veel energie (diesel of gas) kosten. De overbrugging tussen het uitvallen van het electriciteitsnet en het opstarten van de generator wordt opgevangen door een UPS.

