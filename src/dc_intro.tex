Datacentra kunnen de omvang hebben van een bezemkast of enkele voetbalvelden beslaan. Wat ze gemeen hebben is dat het een concentratie is van data verwerkende systemen. Bij de wat kleinere bedrijven worden ze ook vaak aangeduid als serverruimte omdat het de plek is waar de servers staan, de machines die diensten aanbieden aan de gebruikers. Het feit dat wij spreken over datacentra heeft te maken met het feit dat er zich veel meer in een serverruimte bevindt dan alleen servers, er staan bijvoorbeeld ook switches, routers en stroom- en koelvoorzieningen.

Datacentra beginnen bij bedrijven meestal klein en groeien met de organisatie mee. Internet Service Providers\index{ISP}\index{Internet Service Provider} beginnen meestal meteen met een grote ruimte en bedrijven als Google hebben hun datacentra modulair opgebouwd. Zij hebben containers die volledig zelfstandig kunnen werken en zetten er waar nodig containers bij om meer servers te hebben.

Voor al deze vormen van centralisatie van rekencapaciteit zijn een aantal zaken hetzelfde en er zijn ook een aantal verschillen. Dit document geeft je meer inzicht in de technieken die binnen een datacentrum spelen en waar je rekening mee dient te houden.

\url{https://www.dutchdatacenters.nl/datacenters/hoe-werkt-een-datacenter/}
