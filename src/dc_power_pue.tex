PUE of Power Usage Effectiveness is een waarde die gebruikt kan worden om de effectiveit van een datacenter te meten. Naast het verbruik van de servers en netwerkapparatuur is er natuurlijk ook het verbruik door koeling, verlichting, monitoren etc. De PUE geeft aan hoe efficient een datacenter is ingericht. Meestal zit dit getal zo rond de 2 wat betekent dat er 2x zoveel energie het datacenter in gaat als dat er gebruikt wordt door alle servers, switches en routers bij elkaar. Moderne datacenters van Google en Microsoft zitten rond een PUE van 1,2 en zijn daarmee zeer efficient. Een PUE van 1,00 zou betekenen dat alle energie die een datacenter in gaat ook daadwerkelijk gebruikt wordt als computingpower. Dat zou betekenen 100\% efficient en is in de praktijk natuurlijk nooit haalbaar. De PUE wordt berekend door:

\begin{equation}
PUE = \frac{totale\ faciliteitsverbruik}{verbruik\ door\ computerapparatuur}
\end{equation}

Als een ruimte dus 20.000 MW gebruikt en de IT-apparatuur gebruikt 15.000 MW dan is de PUE:

\begin{equation}
PUE = \frac{20000}{15000} = 1,333
\end{equation}

